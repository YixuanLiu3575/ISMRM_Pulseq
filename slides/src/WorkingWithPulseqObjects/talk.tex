% Inbuilt themes in beamer
\documentclass{beamer}

% Theme choice:
\usetheme{Boadilla}

\usepackage[framed]{seqcode}    % Also includes listings packae
\usepackage{verbatim}

\usepackage{tikz}
\usetikzlibrary{calc}

% Title page details: 
\title{Working with Pulseq objects}
%\subtitle{ISMRM virtual meeting: Vendor-Agnostic Pulse Sequence Programming with Pulseq: From Basics to Advanced Topics}
\author{Jon-Fredrik Nielsen}
\date{November 15, 2023}
\logo{\large \LaTeX{}}


\begin{document}

\beamertemplatenavigationsymbolsempty

% Title page frame
\begin{frame}
    \titlepage 
\end{frame}


% Remove logo from the next slides
\logo{}

\lstset{basicstyle=\tiny}     % listing font size


% preliminaries
\begin{frame}[fragile]{Preliminaries}    % need 'fragile' to make lstlisting work

\begin{itemize}

\item Code for this PDF (Latex, MATLAB) available on course Github site

\item Custom functions written for this talk end with '\_\_' suffix. Example:
\vspace{-5mm}
\begin{lstlisting}[language=MATLAB,frame=none]
>> sincrf__;       % creates example Pulseq events ('rf' and 'gz')
\end{lstlisting}

\item Other functions belong to the Pulseq MATLAB toolbox, either as standalone functions
or attached to a 'seq' object. Examples:
\vspace{-5mm}
\begin{lstlisting}[language=MATLAB,frame=none]
>> [rf, gz] = mr.makeSincPulse(pi/2, 'Duration', 2e-3, 'SliceThickness', 5e-3, ...
              'apodization', 0.42, 'timeBwProduct', 4, 'system', sys);
>> seq.plot('timeRange', [0 0.1]);
\end{lstlisting}

\end{itemize}

\end{frame}


% Outline frame
\begin{frame}{Outline}
    \tableofcontents
\end{frame}

%\begin{frame}{TODO}

%* conversions between trapezoids, extended trapezoids and arbitrary gradients,
%issues with the corners of the trapezoid (probably too advanced/in-depth topic
%for the first day).

%\end{frame}


%%%%%%%%%%%%%%%%%%%%%%%%%%%%%%%%%%%%%%%%%%%%%%%%%%%%%%%%%%%%%%%%%%%%%%%%%%%%%%
\section{Units}
%%%%%%%%%%%%%%%%%%%%%%%%%%%%%%%%%%%%%%%%%%%%%%%%%%%%%%%%%%%%%%%%%%%%%%%%%%%%%%
\begin{frame}{Units}

\begin{itemize}
    \item RF
        \begin{itemize}
        \item Unit: Hz
        \item divide by sys.gamma to convert to Tesla
        \end{itemize}
    \item Gradients
        \begin{itemize}
        \item Unit: Hz/m
        \item divide by sys.gamma to convert to Tesla/m
        \end{itemize}
    \item Time in seconds
    \item Flip angles in radians
    \item RF/ADC phase in radians
\end{itemize}

%\begin{center}
%\begin{tabular}{|l|c|c|}
% \hline
%  & Siemens & GE \\ [0.5ex]
% \hline
% Time required to turn A/D converter ON & 10 us & 40 us \\
% \hline
%\end{tabular}
%\end{center}

\end{frame}


%%%%%%%%%%%%%%%%%%%%%%%%%%%%%%%%%%%%%%%%%%%%%%%%%%%%%%%%%%%%%%%%%%%%%%%%%%%%%%
\section{Gradient events}
%%%%%%%%%%%%%%%%%%%%%%%%%%%%%%%%%%%%%%%%%%%%%%%%%%%%%%%%%%%%%%%%%%%%%%%%%%%%%%
\begin{frame}{Gradient events}

3 types:
\begin{itemize}
    \item Trapezoid
    \item Extended trapezoid
    \item Arbitrary gradient
\end{itemize}

\bigskip
Pulseq toolbox defaults
\begin{itemize}
    \item gradRasterTime = 10e-6 s
\end{itemize}

\end{frame}


\begin{frame}{Trapezoid}

Shape specified by rise/flat/fall times

\lstinputlisting[language=MATLAB]{trap.m}

\lstinputlisting[language={},frame=none]{trap.txt}

\tikz[remember picture, overlay] \node[anchor=center] at ($(current page.center)+(2,-1.5)$) {\includegraphics[width=0.4\textwidth]{trap.png}};

\end{frame}



\begin{frame}{Extended trapezoid}

Shape sampled on vertices

\lstinputlisting[language=MATLAB]{exttrap.m}

\lstinputlisting[language={},frame=none]{exttrap.txt}

\tikz[remember picture, overlay] \node[anchor=center] at ($(current page.center)+(2.5,-2.0)$) {\includegraphics[width=0.4\textwidth]{exttrap.png}};

\end{frame}



\begin{frame}{Arbitrary gradient}

Shape sampled on regular raster, on \textcolor{red}{center} of raster periods

\lstinputlisting[language=MATLAB]{arbgrad.m}

\lstinputlisting[language={},frame=none]{arbgrad.txt}

\pause
\tikz[remember picture, overlay] \node[anchor=center] at ($(current page.center)+(3,-1.5)$) {\includegraphics[width=0.5\textwidth]{arbgrad.png}};

\pause
\tikz[remember picture, overlay] \node[anchor=center] at ($(current page.center)+(-3,-1.5)$) {\includegraphics[width=0.5\textwidth]{arbgrad_pulseq.png}};

\end{frame}


\begin{frame}{'first' and 'last'}

\begin{itemize}
    \item Waveform values at beginning of first raster period, and end of last raster period
    \item Calculated by makeArbitraryGradient(), or set manually
    \item Uses: seq.plot();  k-space calculation;
          check for waveform continuity across blocks
    \item not yet saved in .seq file (likely to change in the future)
\end{itemize}

\lstinputlisting[language={},frame=none]{firstlast.txt}

\tikz[remember picture, overlay] \node[anchor=center] at ($(current page.center)+(3.5,-2.1)$) {\includegraphics[width=0.5\textwidth]{firstlast.png}};

\end{frame}


\begin{frame}[fragile]{Modify gradients with scaleGrad()}

\begin{itemize}
    \item Function returns a scaled copy of the input event
    \item Typically used inside scan loop, e.g., 
          to scale the y phase encode gradient:
    \begin{lstlisting}[language=MATLAB,frame=none]
    for j = -Ny/2:Ny/2-1
        seq.addBlock(mr.scaleGrad(gy, j/(Ny/2)));
        ...
    end
    \end{lstlisting}
    \item Opinion: it is good programming practice use scaleGradient() instead of creating a new event from scratch
    \begin{itemize}
        \item Makes the purpose of the gradient event, and relationship to the input gradient, clear  
        \item Scaled events share the same shape ID $\to$ simplifies interpreter code
    \end{itemize}
\end{itemize}

\end{frame}


\begin{frame}[fragile]{Modify gradients with rotate()}

\begin{itemize}
    \item Function returns single-axis gradient events resulting from rotation
    \vspace{-0.5cm}
    \lstinputlisting[language=MATLAB]{rotate.txt}
\end{itemize}

\end{frame}


\begin{frame}[fragile]{Gradient 'surgery' with splitGradientAt()}
\lstinputlisting[language=MATLAB]{splitgradientat.txt}
\end{frame}
\begin{frame}[fragile]{addGradients()}
Returns sum of gradients
\lstinputlisting[language=MATLAB]{addgradients.txt}
\end{frame}



%%%%%%%%%%%%%%%%%%%%%%%%%%%%%%%%%%%%%%%%%%%%%%%%%%%%%%%%%%%%%%%%%%%%%%%%%%%%%%
\section{RF events}
%%%%%%%%%%%%%%%%%%%%%%%%%%%%%%%%%%%%%%%%%%%%%%%%%%%%%%%%%%%%%%%%%%%%%%%%%%%%%%

\begin{frame}{RF events}

\begin{itemize}
    \item Shape sampled on regular raster on center of raster periods, or on vertices
    \item Pulseq toolbox defaults
        \begin{itemize}
            \item rfRasterTime = 1e-6 s
        \end{itemize}
\end{itemize}

\end{frame}


\begin{frame}{RF event: Example 1}

sinc pulse

\vspace{-3mm}
\lstinputlisting[language=MATLAB]{sincrf.m}
\vspace{-0.6cm}
\lstinputlisting[language={},frame=none]{sincrf.txt}

\pause
\tikz[remember picture, overlay] \node[anchor=center] at ($(current page.center)+(2.5,-1)$) {\includegraphics[width=0.7\textwidth]{sincrf.png}};

\end{frame}


\begin{frame}{RF dead time, ringdown time}

\begin{itemize}
    \item Some time is required to turn on RF amplifier
    \begin{itemize}
        \item Varies by vendor (of order $\sim$100~$\mu s$)
    \end{itemize}
    \item Some time is also required to turn off the RF amplifier
    \begin{itemize}
        \item Varies by vendor (of order tens of $\mu s$)
    \end{itemize}
    \item Pulseq toolbox:
    \begin{itemize}
        \item Requires that rf.delay $\geq$ sys.rfDeadTime.  
              makeSincPulse() will insert this delay if needed.
        \item Requires a gap $\geq$ sys.rfRingdownTime between end of RF waveform and end of block. 
              makeSincPulse() will extend the block duration if needed.
    \end{itemize}
    \item Depending on the vendor, it may not always be necessary to specify non-zero dead/ringdown times 
          (more on that later)
\end{itemize}

%\lstinputlisting[language={},frame=none]{sincrf.txt}

\end{frame}


\begin{frame}{RF event: Example 2}

SMS pulse (6 simultaneous slices)

\vspace{-3mm}
\lstinputlisting[language=MATLAB]{arbrf.m}
\vspace{-0.6cm}
\lstinputlisting[language={},frame=none]{arbrf.txt}

\pause
\tikz[remember picture, overlay] \node[anchor=center] at ($(current page.center)+(2.5,-1)$) {\includegraphics[width=0.7\textwidth]{arbrf.png}};

\end{frame}



%%%%%%%%%%%%%%%%%%%%%%%%%%%%%%%%%%%%%%%%%%%%%%%%%%%%%%%%%%%%%%%%%%%%%%%%%%%%%%
\section{ADC events}
%%%%%%%%%%%%%%%%%%%%%%%%%%%%%%%%%%%%%%%%%%%%%%%%%%%%%%%%%%%%%%%%%%%%%%%%%%%%%%
\begin{frame}{ADC events}

\begin{itemize}
    \item Samples assumed to occur at centers of dwell time periods 
    \item The delay defines the timing of the starting edge of the first period
\end{itemize}

\lstinputlisting[language=MATLAB]{makeadc.m}
%\vspace{-0.6cm}
\lstinputlisting[language={},frame=none]{adc.txt}

\end{frame}



\begin{frame}{ADC dead time}

\begin{itemize}
    \item Some time is required to turn on data acquisition board
    \begin{itemize}
        \item Varies by vendor (typically a few tens of $\mu s$)
    \end{itemize}
    \item Pulseq toolbox:
    \begin{itemize}
        \item Requires that adc.delay $\geq$ sys.adcDeadTime.  makeAdc() will insert this delay if needed.
        \item Requires a gap $\geq$ sys.adcDeadTime between end of ADC window and end of block. makeAdc() will extend the block duration if needed.
    \end{itemize}
    \item Depending on the vendor, it may not always be necessary to specify a non-zero dead time before/after the ADC window 
\end{itemize}

\end{frame}



%%%%%%%%%%%%%%%%%%%%%%%%%%%%%%%%%%%%%%%%%%%%%%%%%%%%%%%%%%%%%%%%%%%%%%%%%%%%%%%
\section{Delay events}
%%%%%%%%%%%%%%%%%%%%%%%%%%%%%%%%%%%%%%%%%%%%%%%%%%%%%%%%%%%%%%%%%%%%%%%%%%%%%%%
\begin{frame}{Delay events}

\begin{itemize}
    \item Not real objects: their only function is to (possibly) extend the block duration  
    \item Not stored in the Pulseq file
\end{itemize}

\lstinputlisting[language=MATLAB]{makedelay.m}
%\vspace{-0.6cm}
\lstinputlisting[language={},frame=none]{delay.txt}

\end{frame}


\begin{frame}{Get event duration: mr.calcDuration()}

\begin{itemize}
    \item Example:
\lstinputlisting[language=MATLAB]{calcduration.txt}
    \item Common use: calculate delays for desired TE, TR, etc
\end{itemize}

\end{frame}




%%%%%%%%%%%%%%%%%%%%%%%%%%%%%%%%%%%%%%%%%%%%%%%%%%%%%%%%%%%%%%%%%%%%%%%%%%%%%%
\section{Creating blocks}
%%%%%%%%%%%%%%%%%%%%%%%%%%%%%%%%%%%%%%%%%%%%%%%%%%%%%%%%%%%%%%%%%%%%%%%%%%%%%%

\begin{frame}{Creating blocks}

\lstinputlisting[language=MATLAB]{makeblocks.m}

\lstinputlisting[language={}, frame=none]{block1.txt}

\pause
\tikz[remember picture, overlay] \node[anchor=center] at ($(current page.center)+(2.5,-1.5)$) {\includegraphics[width=0.65\textwidth]{blocks.png}};

\end{frame}


\begin{frame}{Creating blocks}

Rules:
\begin{itemize}
    \item Time is referenced to start of block
    \item Block duration is determined by the longest event
    \item Only one event on each channel (per block)
\end{itemize}

\medskip
To pass seq.checkTiming(), additional rules apply:
\begin{itemize}
    \item Minimum RF delay = sys.rfDeadTime
    \item Minimum RF ringdown time before end of block = sys.rfRingdownTime
    \item Minimum ADC delay = sys.adcDeadTime
    \item Minimum time after ADC event before end of block = sys.adcDeadTime
    \item Many more constraints are verified/checked by checkTiming(),
          which will be covered in the 'Sequence Analysis' talk.
\end{itemize}

\end{frame}



\end{document}
